% Summary index document for the analysis chapter
\chapter{Analyse}
\label{chap:Analyse}

\todo{Intro}
\todo{JAN: Kapitel überarbeiten, Analyse des Problems am Anfang.}

\section{Einführung}
Gemäss der Aufgabenstellung dieser \work{} ist die Programmiersprache für das \tool{} frei wählbar mit der einzigen Voraussetzung, dass man eine \acs{PCAP}-Library einbinden kann. In der Inception-Phase des Projekts ging es daher darum, eine geeignete Sprache und Bibliothek zu wählen.

\section{JNetPcap}
\label{sec:JNetPcap}

todo Jan

\section{Golang und goPacket}
\label{sec:Golang und goPacket}

\subsection{Golang}
Golang, auch Go genannt, ist eine eher junge Programmiersprache seit 2007 von Google Inc. entwickelt wurde. Golang hat einen C-ähnlichen Syntax, bietet aber viele Eigenschaften von modernen Programmiersprachen wie zum Beispiel Garbage Collection, Type-Safety, Dynamic-Typing, Closures und eine grosse Standard-Library.
Im Oktober 2009 wurde die Golang der Öffentlichkeit als Open Source zur Verfügung gestellt.

\subsection{Evaluation}
Unseren ersten Kontakt mit Golang haben wir durch GoProbe von Open Systems gewonnen.
GoProbe erlaubt leichtgewichtiges aggregieren von Paketen und deren effiziente Speicherung. Eine Abfrage der gespeicherten Paketen ist via Querying Flows möglich.

Die Installation von Golang und das Builden von goProbe waren etwas harzig. Es hat aber schlussendlich geklappt. Die Erkenntnisse wie man goProbe erfolgreich installieren kann sind im Anhang dokumentiert.

GoProbe besteht aus drei Modulen. Zum einen ein Modul das selbst goProbe heisst und zum aufzeichnen von Paketen verwendet wird. Die von goProbe aufgezeichneten Pakete werden dann mit goDB gespeichert. GoDB ist eine speziell für Netzwerk-Pakete entwickelte Datenbank. Die gespeicherten Pakete können dann mit goQuery wieder abgefragt werden. Ausserdem steht noch ein optionales Modul namens goConvert zur Verfügung. Die drei Module sind aber ein vollständiges Programm und nicht Bibliotheken die wir einfach so in unser \tool einbinden können. Das von uns entwickelte \tool würde GoProbe als Vorbild nehmen aber wahrscheinlich keine Dependencies darauf haben. In der momentanen Evaluierungs-Phase ist es aber ein gutes Versuchsobjekt für die Performance-Tests.

Golang selbst hat einige sehr angenehme Eigenschaften, so lässt sich der Code z.B. sehr einfach mit dem mitgelieferten godoc dokumentieren. Auch das Installieren von Dependencies via. 'go get package-name' ist sehr hilfreich, vorausgesetzt man hat die GOPATH-Umgebungsvariablen korrekt gesetzt.

\section{Performance Vergleich}
\label{sec:Performance Vergleich}

\subsection{Testaufbau}
Zwei Desktop-Rechner der HSR, mit je 16GB Ram und Intel Xeon 3.4Ghz Quad-Core CPUs, sind via Gigabit-Lan miteinander verbunden. Auf den Rechnern läuft Ubuntu 14.04 x64 sowie jPerf und die jeweils getestete Software.

\begin{figure}[H]
    \begin{center}
		\includegraphics[width=0.7\textwidth]{start/img/PerformanceEvaluation.png}
    \end{center}
    \caption{Aufbau der Testumgebung für Performance Tests mit jPerf}
\end{figure}

\subsection{Testdurchführung}
Auf einem der beiden Rechnern läuft jPerf im Server-Modus sowie die getestete Software. Auf dem anderen Computer läuft jPerf im Client-Modus.
Via. jPerf wird nun soviel Traffic erzeugt um die 1Gbit/s-Leitung möglichst stark auszulasten, d.h. durchschnittlich 900mbit/s. Die getestete Software zeichnet dabei die ganzen Pakete auf. Gemäss der \osag{} sind mit Lastspitzen von bis zu 300mbit/s zu rechnen.

\subsection{Ergebnisse}
Sowohl mit JNetPcap (Java) als auch mit goProbe (Golang) lassen sich mehr als 300mbit/s an Verkehr aufzeichnen. GoProbe ist mit den Durchschnittlich 16\% CPU Auslastung aber etwas performanter als JNetPcap. Die 31\% CPU Lastspitze beim Java Programm gibt es jeweils nur, wenn das Programm zum ersten Mal gestartet wird. Dies kommt daher, weil zuerst eine \acs{JVM} hochgefahren werden muss.
Beim Speicherverbrauch hat goProbe aber deutlich die Nase vorne. So wird tatsächlich nur ein Bruchteil des physischen Speichers (RES\footnotemark[2]) gegenüber Java verwendet.

\begin{table}[h]
\begin{tabular}{l|l|l|l|l|l|l}
\textbf{Software} & \textbf{CPU Top} & \textbf{CPU Ø} & \textbf{Mem Ø} & \textbf{VIRT\footnotemark[1]} & \textbf{RES\footnotemark[2]} & \textbf{SHR\footnotemark[3]} \\ \hline
JNetPcap          & 31\%             & 20\%           & 0.9\%          & 7030296kb       & 147840kb       & 19588kb        \\ \hline
goProbe           & 18\%             & 16\%           & 0.3\%          & 315268kb        & 1964kb         & 1676kb         \\
\end{tabular}
\caption{Performance Vergleich: jNetPcap vs. goProbe}
\end{table}

Die oben dargestellten Ergebnisse wurden mit der Applikation \code{Top} aufgezeichnet. \code{Top} zeigt eine Echtzeitansicht des laufenden Systems durch eine Liste von Tasks, die momentan vom Linux Kernel verwaltet werden\cite[:12]{ubuntu_top}.

\footnotetext[1]{VIRT steht für die virtuelle Grösse eines Prozesses.}
\footnotetext[2]{RES steht für den tatsächlich, physisch verbrauchten Hauptspeicher.}
\footnotetext[3]{SHR zeigt wie viel von VIRT mit anderen Prozessen teilbar ist. Dazu gehören z.B. Shared Libraries.}

\section{Entscheidung}
Open Systems AG würde es bevorzugen, wenn Go statt Java verwendet wird. Die Ergebnisse des Performance-Tests sprechen ebenfalls für Go. Und wir haben durchaus auch das Interesse  eine neue Programmiersprache zu lernen.
In Anbetracht dessen haben wir uns entschieden, das \tool{} mit Go zu entwickeln.

\section{ESP Aufbau}
\label{sec:ESP Aufbau}

\noindent Encapsulating Security Payload (ESP) wird bei IPsec (VPN) eingesetzt. Es gewährleistet die Vertraulichkeit und Integrität von Paketen und kümmert sich um die Authentisierung. Durch diese Integritätssicherung werden Pakete vor Manipulation geschützt.ESP verschlüsselt, im Unterschied zu Authentication Header (AH), die Nutzdaten. Bei AH werden nur die Integrität und Echtheit sichergestellt.\cite{elektronik_kompendium}

\noindent Der ESP header wird zwischen dem IP header und dem darunterliegenden Protokoll eingefügt (transport mode), oder es kapselt das ganze IP-Paket (tunnel mode).\cite{rfc4303}

\noindent Der tunnel mode wird vor allem bei der Verbindung zwischen zwei Netzwerken über eine unsichere Verbindung eingesetzt. Der Modus unterstützt aber prinzipiell alle Arten von VPN-Anwendungen.Bei dieser Verwendung wird das ganze IP-Paket verschlüsselt und in ein neues IP-Paket verpackt. So wird das gesamte Paket durch ESP abgeschirmt und die eigentliche IP-Adresse des Absenders ist nicht mehr ersichtlich.Diese Methode hat natürlich einen gewissen Overhead. Es kommen 8 Byte für den ESP-Header, 16-20 Byte ESP-Trailer und für den neuen IP-Header 20 Byte hinzu.\cite{elektronik_kompendium}

\begin{figure}[ht]
    \begin{center}
        \includegraphics[trim=1 0 0 0,clip,width=\textwidth]{mainpart/analyse/img/ESP_Tunnelmode.png}
    \end{center}
    \caption{Aufbau eines Pakets im Tunnelmode}
    %\label{fig:AST_function_call_with_and_without_template_id}
\end{figure}

\noindent Bei einer Situation in der nur zwei Rechner miteinander verbunden werden, kann der Transportmodus verwendet werden. Dieser Modus unterstützt nur Host-to-Host Verbindungen. Da es für IPsec nicht unbedingt notwendig ist IP-Pakete vollständig neu zu enkapeln, können beim transport mode der originale IP-Header verwendet werden.Es kommen 8 Byte ESP-Header und 16-20Byte ESP-Trailer hinzu. Damit wird der Overhead kleiner, da man keinen zusätzlichen IP-Header benötigt.\cite{elektronik_kompendium}

\begin{figure}[ht]
    \begin{center}
        \includegraphics[trim=1 0 0 0,clip,width=\textwidth]{mainpart/analyse/img/ESP_Transportmode.png}
    \end{center}
    \caption{Aufbau eines Pakets im Transportmode}
    %\label{fig:AST_function_call_with_and_without_template_id}
\end{figure}


\noindent Wichtige Felder

\begin{table}[H]
\begin{tabularx}{\textwidth}{l|>{\raggedright\arraybackslash}X} 
        \hline
        Security Parameters\\ Index (SPI) 32bits & Dieser zufällig festgelegte Wert in Kombination mit der IP-Adresse des Ziels wird für eine Identifikation der Verbindung benötigt. Bei jeder neuen Verbindung wird die SPI neu gesetzt.                                                                                                                                                                                                                                                                         \\ \hline
        Sequence number 32bits                 & Die Sequence number wird für jedes Paket gesetzt und wird danach für jedes neue Paket um 1 erhöht. Bei einer neuen Verbindung wird die Sequece number stets auf 1 gesetzt.Falls Anti-Replay eingesetzt wird (standardmässig aktiviert) darf sich die Sequence number nicht wiederholen. Daher wird bevor das 2\^{}32 paket gesendet wird die Verbindung zurückgesetzt und eine neue SPI ausgehandelt. Damit wird auch die Sequence number wieder zurückgesetzt. \\
        \hline
\end{tabularx}
\caption{Wichtige ESP Felder}
\end{table}

\cleardoublepage
\section{MTU Discovery}

Netzwerk Pakete beinhalten typischerweise mehrere Netzwerkprotokolle. Diese Protokolle sind in Layern angeordnet und haben klar definierte Schnittstellen. Dadurch sind sie grösstenteils unabhängig voneinander und können ausgetauscht werden. Bei der Kommunikation mit einem Webserver hat man zum Beispiel den folgenden Aufbau\footnotemark[1]. HTTP über TCP/IP über Ethernet.

\begin{figure}[H]
    \begin{center}
        \includegraphics[trim=1 0 0 0,clip,width=\textwidth]{mainpart/analyse/img/HTTP_Stack}
    \end{center}
    \caption{Protokoll Stack eines typischen HTTP Requests}
\end{figure}

\footnotetext[1]{Bild: HSR Vorlesung CN1 - Steffen/Stettler, 29.07.2014, 1-Grundlagen.ppt}

Man ist aber nicht an genau diese Struktur gebunden. Man könnte zum Beispiel den Ethernet Layer durch einen anderen Layer austauschen und die restlichen Protokolle so belassen.

Um das Zusammenspiel der unterschiedlichen Netzwerk Layern besser zu veranschaulichen wurde das \ac{OSI}-Modell entwickelt. Das OSI Modell besteht aus 7 Layern die jeweils unterschiedliche Aufgaben haben. Protokolle die im gleichen \ac{OSI}-Layer mit klaren Schnittstellen definiert wurde sind einfach austauschbar.

\begin{figure}[H]
    \begin{center}
        \includegraphics[trim=1 0 0 0,clip,width=\textwidth]{mainpart/analyse/img/OSI_Modell}
    \end{center}
    \caption{\ac{OSI} Referenzmodell}
\end{figure}

Für die \ac{MTU} Discovery sind vor allem die Layer 1,2,3 des \ac{OSI}-Modells\footnotemark[1] wichtig. In Layer 1 \& 2 findet die eigentliche physische Übertragung statt. So ist beispielsweise das Ethernet Protokoll sowohl im physischen Layer als auch im Data-Link-Layer vorhanden.

Die Protokolle der ersten beiden Layern haben im Vergleich zu den höheren Layern einen grossen Unterschied. Sie können nur eine begrenzt grosse Payload enthalten. So ist die maximale Payload bei Ethernet zum Beispiel 1500 Bytes. Ein Paket des Netzwerk-Layers (Layer 3) darf als maximal 1500 Bytes gross sein wenn es über Ethernet übertragen werden soll. Diese maximale Payload nennt man \acl{MTU} \acs{MTU}.

\footnotetext[1]{Bild: HSR Vorlesung CN1 - Steffen/Stettler, 29.07.2014, 1-Grundlagen.ppt}

Wenn ein Layer 3 Paket grösser als die \ac{MTU} ist dann muss es in mehrere Pakete aufgeteilt werden. Dieser Vorgang wird Fragmentierung genannt. Auch wenn Paket-Fragmentierung auftritt funktioniert eine Netzwerkverbindung normalerweise immer noch problemlos, es gibt jedoch auf Grund der Fragmentierung Performance-Einbussen. Daher versucht man, wenn möglich, Paket-Fragmentierung zu vermeiden.

\subsection{Hardware Abhängigkeit}
Die \ac{MTU} ist ein Hardware abhängiger Wert, der je nach der eingesetzten Technologie anders ist. Die Tabelle unten zeigt die \ac{MTU}s von einigen heute verwendeten Übertragungstechnologien.

\begin{table}[H]
\begin{tabularx}{\textwidth}{l|>{\raggedright\arraybackslash}X} 
\textbf{Technologie} & \textbf{MTU in Bytes} \\
\hline
\ac{FDDI} \cite[:915]{rfc1191} & 4352 \\
Ethernet \cite[:915]{rfc1191} & 1500 \\
\ac{PPPoE} \cite[:374]{rfc2516}& 1492 \\
X.25 Networks / ISDN \cite[:915]{rfc1191} & 576 \\
\end{tabularx}
\caption{Typische MTU-Grössen}
\end{table}

Wenn eine Verbindung von A nach B in mehrere Wegstrecken aufgeteilt ist und für diese Wegstrecken unterschiedliche Technologien verwendet werden dann ist die tiefste \ac{MTU} relevant für das Versenden von Paketen.

\subsection{Path MTU Discovery}
Da die \ac{MTU} bei einer neuen Verbindung noch unbekannt ist muss sie zuerst ermittelt werden. Normalerweise geschieht dies automatisch via \ac{PMTUD}. Dabei werden \ac{ICMP} Pakete unterschiedlicher Grösse die mit einem \enquote{Don't fragment} Flag versehen sind über die Verbindung gesandt. Wenn ein solches Paket auf ein Netzwerkgerät trifft dass nur eine tiefere \ac{MTU} unterstützt wird das Paket nicht weitergesendet, stattdessen wird ein \ac{ICMP} Paket mit dem Inhalt "Fragmentation needed" retourniert. So weiss der \ac{PMTUD} Algorithmus dass die \ac{MTU} der Verbindung überschritten wurde \cite[:131]{rfc1191}.

\ac{PMTUD} hat jedoch ein Problem. Router die, die \ac{ICMP} Pakete weiterleiten oder aber ein \enquote{Fragmentation needed} Paket zurücksenden sollten tun dies nicht immer. Dieses Fehlverhalten gibt es aus mehreren Gründen. Zum einen wegen Kernel-Bugs, Fehlkonfigurationen und zum anderen auch weil Firewalls manchmal so konfiguriert werden dass sie \ac{ICMP} Nachrichten nicht durchlassen auf Grund von Sicherheitsbedenken \cite[:137]{rfc2923}.

Da man sich also nicht auf \ac{PMTUD} verlassen kann um die \ac{MTU} einer Verbindung festzustellen wird mit dieser Arbeit eine \ac{MTU} Discovery implementiert die innerhalb einer \ac{IPsec} \ac{VPN} Verbindung durchgeführt werden kann.
