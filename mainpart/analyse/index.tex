% Summary index document for the analysis chapter
\chapter{Analyse}
\label{chap:Analyse}

In diesem Kapitel wird unsere Wahl der Pcap-Bibliothek erläutert.
In diesem Kapitel wird unsere Wahl der Programmiersprache und \acs{PCAP}-Library erläutert.

<<<<<<< HEAD
\section{Vergleich von Java und Go}
\label{sec:Vergleich von Java und Go}

\subsection{Performancetest}
Die beiden Sprachen wurden mithilfe einer Testumgebung verglichen um Performanceunterschiede festzustellen. Die Testumgebung wurde folgendermassen aufgebaut.

\includegraphics[width=0.7\textwidth]{start/img/PerformanceEvaluation.png}

Der Datenverkehr wurde mithilfe von Jperf simuliert und jeweil mit den beiden Testapplikationen aufgezeichnet.\\
Für Java wurde eine Testapplikation mit der Library JnetPcap erstellt. JnetPcap ist eine Opensource Library die einen wrapper für die Libpcap-Library zur verfügung stellt.\\
Für Go wurden die Pakete mithilfe von GoProbe aufgezeichnet. GoProbe ist ebenfalls Opensource und verwendet die Libpcap-Library.

\subsection{Performancetest Resultat}
Auslastung des Prozessors...

=======
\section{Einführung}
Gemäss der Aufgabenstellung dieser \work ist die Programmiersprache für das \tool frei wählbar mit der einzigen Voraussetzung dass man eine \acs{PCAP}-Library einbinden kann. In der Inception-Phase des Projekts ging es daher darum eine geeignete Sprache und Bibliothek zu wählen.

\section{JNetPcap}
\label{sec:JNetPcap}

todo Jan

\section{Golang und goPacket}
\label{sec:Golang und goPacket}

\subsection{Golang}
Golang, auch Go genannt, ist eine eher junge Programmiersprache seit 2007 von Google Inc. entwickelt wurde. Golang hat einen C-ähnlichen Syntax, bietet aber viele Eigenschaften von modernen Programmiersprachen wie zum Beispiel Garbage Collection, Type-Safety, Dynamic-Typing, Closures und eine grosse Standard-Library.
Im Oktober 2009 wurde die Golang der Öffentlichkeit als Open Source zur Verfügung gestellt.

\subsection{Evaluation}
Unseren ersten Kontakt mit Golang haben wir durch GoProbe von Open Systems gewonnen.
GoProbe erlaubt leichtgewichtiges aggregieren von Paketen und deren effiziente Speicherung. Eine Abfrage der gespeicherten Paketen ist via Querying Flows möglich.

Die Installation von Golang und das Builden von goProbe waren etwas harzig. Es hat aber schlussendlich geklappt. Die Erkenntnisse wie man goProbe erfolgreich installieren kann sind im Anhang dokumentiert.

GoProbe besteht aus drei Modulen. Zum einen ein Modul das selbst goProbe heisst und zum aufzeichnen von Paketen verwendet wird. Die von goProbe aufgezeichneten Pakete werden dann mit goDB gespeichert. GoDB ist eine speziell für Netzwerk-Pakete entwickelte Datenbank. Die gespeicherten Pakete können dann mit goQuery wieder abgefragt werden. Ausserdem steht noch ein optionales Modul namens goConvert zur Verfügung. Die drei Module sind aber ein vollständiges Programm und nicht Bibliotheken die wir einfach so in unser \tool einbinden können. Das von uns entwickelte \tool würde GoProbe als Vorbild nehmen aber wahrscheinlich keine Dependencies darauf haben. In der momentanen Evaluierungs-Phase ist es aber ein gutes Versuchsobjekt für die Performance-Tests.

Golang selbst hat einige sehr angenehme Eigenschaften, so lässt sich der Code z.B. sehr einfach mit dem mitgelieferten godoc dokumentieren. Auch das Installieren von Dependencies via. 'go get package-name' ist sehr hilfreich, vorausgesetzt man hat die GOPATH-Umgebungsvariablen korrekt gesetzt.

\section{Performance Vergleich}
\label{sec:Performance Vergleich}

\subsection{Testaufbau}
Zwei Desktop-Rechner der HSR, mit je 16GB Ram und Intel Xeon 3.4Ghz Quad-Core CPUs, sind via Gigabit-Lan miteinander verbunden. Auf den Rechnern läuft Ubuntu 14.04 x64 sowie jPerf und die jeweils getestete Software.

\begin{figure}[H]
    \begin{center}
		\includegraphics[width=0.7\textwidth]{start/img/PerformanceEvaluation.png}
    \end{center}
    \caption{Aufbau der Testumgebung für Performance Tests mit jPerf}
\end{figure}

\subsection{Testdurchführung}
Auf einem der beiden Rechnern läuft jPerf im Server-Modus sowie die getestete Software. Auf dem anderen Computer läuft jPerf im Client-Modus.
Via. jPerf wird nun soviel Traffic erzeugt um die 1Gbit/s-Leitung möglichst stark auszulasten, d.h. durchschnittlich 900mbit/s. Die getestete Software zeichnet dabei die ganzen Pakete auf. Gemäss der \osag{} sind mit Lastspitzen von bis zu 300mbit/s zu rechnen.

\subsection{Ergebnisse}
Sowohl mit JNetPcap (Java) als auch mit goProbe (Golang) lassen sich mehr als 300mbit/s an Verkehr aufzeichnen. GoProbe ist mit den Durchschnittlich 16\% CPU Auslastung aber etwas performanter als JNetPcap. Die 31\% CPU Lastspitze beim Java Programm gibt es jeweils nur, wenn das Programm zum ersten Mal gestartet wird. Dies kommt daher, weil zuerst eine \acs{JVM} hochgefahren werden muss.
Beim Speicherverbrauch hat goProbe aber deutlich die Nase vorne. So wird tatsächlich nur ein Bruchteil des physischen Speichers (RES\footnotemark[2]) gegenüber Java verwendet.

\begin{table}[h]
\begin{tabular}{l|l|l|l|l|l|l}
\textbf{Software} & \textbf{CPU Top} & \textbf{CPU Ø} & \textbf{Mem Ø} & \textbf{VIRT\footnotemark[1]} & \textbf{RES\footnotemark[2]} & \textbf{SHR\footnotemark[3]} \\ \hline
JNetPcap          & 31\%             & 20\%           & 0.9\%          & 7030296kb       & 147840kb       & 19588kb        \\ \hline
goProbe           & 18\%             & 16\%           & 0.3\%          & 315268kb        & 1964kb         & 1676kb         \\
\end{tabular}
\caption{Performance Vergleich: jNetPcap vs. goProbe}
\end{table}

Die oben dargestellten Ergebnisse wurden mit der Applikation \code{Top} aufgezeichnet. \code{Top} zeigt eine Echtzeitansicht des laufenden Systems durch eine Liste von Tasks, die momentan vom Linux Kernel verwaltet werden\cite[:12]{ubuntu_top}.

\footnotetext[1]{VIRT steht für die virtuelle Grösse eines Prozesses.}
\footnotetext[2]{RES steht für den tatsächlich, physisch verbrauchten Hauptspeicher.}
\footnotetext[3]{SHR zeigt wie viel von VIRT mit anderen Prozessen teilbar ist. Dazu gehören z.B. Shared Libraries.}

\section{Entscheidung}
Open Systems AG würde es bevorzugen wenn wir Golang statt Java einsetzen. Die Ergebnisse des Performance-Tests sprechen ebenfalls für Golang. Und wir haben durchaus auch das Interesse einmal eine neue Programmiersprache zu lernen.
In Anbetracht dessen haben wir uns entschieden das \tool mit Golang zu entwickeln.
>>>>>>> origin/master
