\section{Funktionale Anforderungen}
\label{sec:Funktionale Anforderungen}

\subsection{Muss-Kriterien}

\begin{itemize}

\item \textbf{M0: Paket Verlust messen} \\
Das \tool{} soll den Paket-Verlust von allen konfigurierten \ac{IPsec} \ac{VPN}-Verbindungen bestimmen.

\item \textbf{M1: Paket Verlust melden} \\
Wenn ein konfigurierter Grenzwert überschritten wird soll das \tool{} automatisch eine Meldung an einen zentralen Syslog Server absetzen.

\item \textbf{M2: Paket Verluste dokumentieren} \\
Für jedes verlorene Paket soll in einem lokalen Log ein Eintrag erzeugt werden. Wenn das Paket später noch ankommt soll dies ebenfalls vermerkt werden.

\item \textbf{M3: MTU Bestimmen} \\
Das \tool{} soll in der Lage sein die \ac{MTU} zwischen zwei Routern bestimmen zu können. Dabei läuft das \tool{} auf mindestens einem der beiden Routern.

\item \textbf{M4: MTU Melden} \\
Eine erfolgreich bestimmte \ac{MTU} soll an einen zentralen Syslog Server gemeldet werden.

\item \textbf{M6: Konfiguration} \\
Alle notwendigen Einstellungen lassen sich vom User ohne Änderung des Source Codes vornehmen.

\end{itemize}

\subsection{Soll-Kriterien}
\begin{itemize}

\item \textbf{S0: MTU für mehrere Tunnels bestimmen:} \\
Für die \ac{MTU} Bestimmung sind mehrere Tunnels konfigurierbar. Wenn die \ac{MTU} Bestimmung ausgeführt wird, werden die \ac{MTU}s von allen konfigurierten Tunnels bestimmt.

\item \textbf{S1: Parallele MTU Bestimmung} \\
Die \ac{MTU} Bestimmung für mehrere Tunnels soll parallel ablaufen um Zeit zu sparen.

\item \textbf{S2: Paket Verlust Messung für mehrere Tunnels} \\
Das \tool{} soll in der Lage sein den Paket Verlust für alle Tunnels, die auf dem Rechner laufen zu bestimmen.
  	
\item \textbf{S3: Daemon Modus} \\
Um 24/7 den Paket-Verlust messen zu können soll das Tool in der Lage sein als Unix-Daemon zu laufen.

\end{itemize}

\subsection{Kann-Kriterien}
\begin{itemize}

\item \textbf{K0: Konfiguration automatisch auslesen:} \\
Die Applikation liest die benötigte Konfiguration aus vorhandenen Konfigurationsfile für die \acs{IPsec} Tunnels aus. So muss weniger für das \tool selbst konfiguriert werden.

\item \textbf{K1: Statistiken erstellen:} \\
Es kann direkt aus dem \tool eine Statistik zum Paket Verlust erstellt werden.

\end{itemize}