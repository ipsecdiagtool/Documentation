\section{Nichtfunktionale Anforderungen}
\label{sec:Nichtfunktionale Anforderungen}

\subsection{Angemessenheit}
\begin{itemize}
\item Es sollen alle der genannten Muss \& Soll Kriterien erfüllt werden.
\item Alle in den Use Cases beschriebenen Anwendungsfälle sollen durch den User des \tool{}s ausgeführt werden können.
\item Die Applikation soll schlank und in einem einheitlichen Stil programmiert werden. Zudem sollen die Vorschriften der Codeformatierung von Go übernommen werden. Dadurch wird eine höhere Übersichtlichkeit und leichtere Einarbeitung in den Code gewährleistet.
\end{itemize}

\subsubsection{Funktionalität}
\begin{itemize}
\item Das Window für die Gültigkeit von \ac{ESP} Paketen ist konfigurierbar. Die Konfiguration führt zu einer individuelleren Erkennung von Paketverlusten. So kann auch bestimmt werden ab welcher Abweichung verspätetes Paket als verloren angesehen wird.
\item Das Diagnose Tool funktioniert mindestens bis zu einer Datenrate von 300Mbit/s.
\item Es ist konfigurierbar welches Netzwerkinterface bei der Überwachung verwendet werden soll.
\end{itemize}

\subsubsection{Testbarkeit}
\begin{itemize}
\item Die Businesslogik soll über alle Use Cases (Normal Verhalten) mit automatischen Unit-Tests abgedeckt sein.
\item Für die Unit Tests soll die "testing" Package von Go verwendet werden. Mithilfe dieser Package lassen sich Unit Tests schreiben die automatisch ausgeführt werden und so bei jeder Änderung Feedback zur Integrität der Applikation geben.
\end{itemize}

\subsubsection{Fehlertoleranz}
\begin{itemize}
\item Das System muss auch nach einem Neustart in einem definierten Zustand sein. Dies auch bei einem unvorhergesehenen Abbruch des Programms. 
\item Um eine Fehlerhafte Ausführung und damit verbundene Schäden zu vermeiden, wird vor dem Start überprüft ob die eingestellte Konfiguration gültig ist.
\end{itemize}

\subsubsection{Änderbarkeit}
\begin{itemize}
\item Konfiguration, Daten und Reports werden in einem Standard gespeichert, welcher auch von anderen Systemen gelesen werden kann (z.B. CSV, XML, JSON).
\item Der Code inklusive Kommentar sind auf Englisch um eine leichtere Anpassbarkeit zu erreichen.
\end{itemize}

\subsubsection{Anforderungen an Umgebung}
\begin{itemize}
\item Das Tool ist unter dem Betriebssystem Linux lauffähig, es werden jedoch Root-Rechte benötigt.
\item Für die Ausführung des \tool{}s ist eine Installation der \ac{PCAP}-Library libpcap notwendig.
\item Für das Kompilieren des \tool{}s ist eine Go-Entwicklungsumgebung notwendig.
\item Das \tool{} ist Open Source und der Quellcode wird frei verfügbar unter der MTI Lizenz veröffentlicht.
\end{itemize}