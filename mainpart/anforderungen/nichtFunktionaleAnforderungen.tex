\section{Nichtfunktionale Anforderungen}
\label{sec:Nichtfunktionale Anforderungen}

\subsection{Angemessenheit}
\begin{itemize}
\item Es sollen alle genannten Muss \& Soll Kriterien erfüllt werden.
\item Alle in den Use Cases beschriebenen Anwendungsfälle sollen durch den User des \tool{}s ausgeführt werden können.
\item Die Applikation soll schlank und in einem einheitlichen Stil programmiert werden. Der Source Code soll gemäss den Vorschriften der Programmiersprache formatiert werden.
\end{itemize}

\subsubsection{Funktionalität}
\begin{itemize}
\item Das Window für die Gültigkeit von \ac{ESP} Paketen ist konfigurierbar. Dies erlaubt eine individuelle Erkennung von Paket-Verlusten. So kann auch bestimmt werden ab welcher Abweichung ein verspätetes Paket als verloren angesehen wird.
\item Das \tool{} funktioniert mindestens bis zu einer Datenrate von 300Mbit/s.
\item Es ist konfigurierbar welches Netzwerkinterface bei der Überwachung verwendet werden soll.
\end{itemize}

\subsubsection{Testbarkeit}
\begin{itemize}
\item Alle Use Cases sollen über Unit Tests abgedeckt werden.
\item Der Build Server soll so konfiguriert werden, dass die Unit Tests jeweils nach jeder Kompilierung automatisch ausgeführt werden.
\end{itemize}

\subsubsection{Fehlertoleranz}
\begin{itemize}
\item Das System muss auch nach einem Neustart in einem definierten Zustand sein. Dies muss auch bei einem unvorhergesehenen Abbruch des Programms gewährleistet sein. 
\item Um eine Fehlerhafte Ausführung und damit verbundene Schäden zu vermeiden, wird vor dem Start überprüft, ob die eingestellte Konfiguration gültig ist.
\end{itemize}

\subsubsection{Änderbarkeit}
\begin{itemize}
\item Konfiguration, Daten und Reports werden in einem Standard gespeichert, welcher auch von anderen Systemen gelesen werden kann (z.B. CSV, XML, JSON).
\item Alle Namen und Kommentare im Source Code werden auf Englisch verfasst.
\end{itemize}

\subsubsection{Anforderungen an Umgebung}
\begin{itemize}
\item Das Tool ist unter dem Betriebssystem Linux lauffähig, es werden jedoch Root-Rechte benötigt.
\item Für die Ausführung des \tool{}s ist eine Installation der \ac{PCAP}-Library libpcap notwendig.
\item Für das Kompilieren des \tool{}s ist eine Go-Entwicklungsumgebung notwendig.
\end{itemize}