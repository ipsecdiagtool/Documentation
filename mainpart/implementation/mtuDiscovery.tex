\section{MTU Discovery}
\label{sec:MTU Discovery}

Die \ac{MTU} Discovery wurde in der \code{mtu} Package implementiert. Darin enthalten sind \code{analyze.go}, \code{capture.go} und \code{send.go} sowie die dazugehörigen Unit Tests.

\subsection{Öffentliche Funktionen}
Die \code{mtu} Package bietet die folgenden öffentlichen Funktionen:

\begin{itemize}
\item \textbf{Init(config config.Config, icmpPackets chan gopacket.Packet):} \\
Diese Funktion wird zur Initialisierung der \code{mtu} Package verwendet. Die Package muss zwingend vor der Verwendung der \code{FindAll()} Funktion initialisiert werden. Zur Initialisierung wird die Konfiguration sowie die der für \ac{ICMP} Pakete verwendete Go-Channel übergeben.
\item \textbf{FindAll():} \\
\code{FindAll()} sucht die \ac{MTU}s aller konfigurierten Verbindungen und meldet die Ergebnisse via der \code{logging} Package.
\item \textbf{RequestDaemonMTU(appID int, sourceIP string, destinationIP string):} \\
Die Funktion \code{RequestDaemonMTU(..)} sendet ein ICMP Request an die übergebene Destionation IP mit dem Kommando einen \ac{MTU} Discovery Vorgang zu starten. Die Funktion erwartet die AppID, Source- und Destination Adresse.
\end{itemize}

\subsection{Implementation des FastMTU Algorithmus}
Der FastMTU Algorithmus ist in \code{analyze.go} implementiert. Bevor der \ac{MTU} Discovery Vorgang gestartet werden kann muss die \code{mtu} Package via der \code{mtu.init(..)} Funktion initialisiert werden. Bei der Initialisierung wird Konfiguration übergeben sowie ein Go-Channel der abgefangene \ac{ICMP} Pakete liefert (ICMP-Channel). Ausserdem wird die Funktion \code{mtu.handlePackets(..)} in einer Goroutine gestartet. Diese Funktion behandelt die über den ICMP-Channel hereinkommenden Pakete.

\begin{lstlisting}[language=go, caption=mtu.Init(..) Funktion]                    
func Init(config config.Config, icmpPackets chan gopacket.Packet) {
	conf = config
	icmpPacketsStage1 = icmpPackets
	icmpPacketsStage2 = make(chan gopacket.Packet, 100)
	initalized = true
	go handlePackets(icmpPacketsStage1, icmpPacketsStage2, conf.ApplicationID)
}
\end{lstlisting}

Nachdem die Package initialisiert ist kann der \ac{MTU} Discovery Vorgang durch einen Aufruft der \code{mtu.FindAll()} Funktion gestartet werden. Die \code{mtu.FindAll()} Funktion iteriert nun durch alle konfigurierten Tunnels und startet für jeden Tunnel eine Goroutine (\code{mtu.find(..)}), welche die spezifische Suche nach der \ac{MTU} durchführt.
Ausserdem wird noch eine Goroutine (\code{mtu.distributeMtuOkPackets(..)}) gestartet welche die Verteilung der hereinkommenden \ac{ICMP} Pakete auf die \code{mtu.find(..)}-Goroutinen übernimmt.
Danach wartet die \code{mtu.FindAll()} Funktion auf die Rückmeldungen der \code{mtu.Find(..)}-Goroutinen.

\begin{lstlisting}[language=go, caption=mtu.Init(..) Funktion]                    
func FindAll() {
	if !initalized { log.Println("Please init...") return }
	c := conf
	//Setup a mtuOK channel for each config
	var mtuOkChannels = make(map[int]chan int)
	for conf := range c.MTUConfList {
		mtuOkChannels[conf] = make(chan int, 100)
	}
	var quitDistribute = make(chan bool)
	go distributeMtuOkPackets(icmpPacketsStage2, mtuOkChannels, quitDistribute)

	var wg sync.WaitGroup
	for conf := range c.MTUConfList {
		logging.InfoLog("Starting MTU Disc...")
		wg.Add(1)
		go find(c.MTUConfList[conf], c.ApplicationID,
			conf, mtuOkChannels[conf], &wg)
	}
	//Wait until all MTUs have been detected
	wg.Wait()
	quitDistribute <- true
}
\end{lstlisting}



\subsection{ICMP Paket Payload}
Alle vom \tool{} verschickten \acs{ICMP} Pakete haben jeweils die folgende Payload:

\begin{itemize}
  \item \textbf{AppID:} Eindeutige ID des aktiven \tool{}. Wird verwendet um mehrere gleichzeitig laufende \tool{} auf einem Rechner zu unterscheiden. Die AppID wird entweder in der Konfiguration fix festgelegt oder auf 0 gesetzt. Wenn die AppID in der Konfiguration auf 0 gesetzt wurde dann wird beim Programmstart eine zufällige AppID generiert.
  \item \textbf{ChannelID:} ID des GO-Channels von dem das Paket versendet wurde. Wird benötigt um für mehrere Tunnels gleichzeitig die \acs{MTU} festzustellen zu können. Die ChannelID wird jeweils beim Start eines \acs{MTU} Discovery Vorgangs zugeteilt.
  \item \textbf{Command:} Die eigentliche Nachricht des Pakets. Wird verwendet um sicherzustellen dass dieses \acs{ICMP} Paket wirklich von einem \tool{} versendet wurde und nicht ein sonstiges \acs{ICMP} Paket.
  \item \textbf{Null-Array:} Ein mit Nullen gefülltes Array von variabler Grösse. Wird verwendet um dem Paket seine vorbestimmte Grösse zu geben.
\end{itemize}

