\section{Risikomanagment}
\label{sec:Risikomanagment}

\subsection{Projektspezifisch}

\begin{table}[h]
\centering
\resizebox{\textwidth}{!}{%
\begin{tabular}{|l|l|}
\hline
\multicolumn{2}{|l|}{\cellcolor[HTML]{C0C0C0}\textbf{R1: Performance der PCAP Library reicht nicht}}                                                                                                                              \\ \hline
Beschreibung             & \begin{tabular}[c]{@{}l@{}}Die PCAP Library ist nicht genügend performant um alle Pakete aufzuzeichnen,\\ es kommt zu Paket-Verlusten die nichts mit der Verbindung zu tun haben.\end{tabular}         \\ \hline
Massnahme                & \begin{tabular}[c]{@{}l@{}}Wir führen bereits vor der definitiven Wahl der Library Performance Tests durch\\ und können so feststellen ob sie unseren Ansprüchen (300mbit/s peak) genügt.\end{tabular} \\ \hline
Vorgehen beim Eintreffen & Eine andere PCAP Library wählen.                                                                                                                                                                       \\ \hline
\end{tabular}
}
\end{table}

\begin{table}[h]
\centering
\resizebox{\textwidth}{!}{%
\begin{tabular}{|l|l|}
\hline
\multicolumn{2}{|l|}{\cellcolor[HTML]{C0C0C0}\textbf{R2: Java kann nicht verwendet werden}}                                                                                                                                                          \\ \hline
Beschreibung             & \begin{tabular}[c]{@{}l@{}}Die gewählt Programmiersprache ist nicht genügend performant, siehe Risiko 1,\\ oder sie darf aus sonstigen Gründen nicht verwendet werden.\end{tabular}                                       \\ \hline
Massnahme                & \begin{tabular}[c]{@{}l@{}}Die Wahl der Programmiersprache wird mit dem Kunden möglichst früh besprochen. \\ Mögliche Performance-Probleme werden wie bei Risiko 1 durch einen Performance\\ Test abgeklärt.\end{tabular} \\ \hline
Vorgehen beim Eintreffen & Eine andere Programmiersprache und eine andere PCAP Library/Wrapper wählen.                                                                                                                                               \\ \hline
\end{tabular}
}
\end{table}
\subsection{Allgemeine}
todo

\subsection{Risikoschätzung}
todo
