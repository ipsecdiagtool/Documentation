\section{Risikomanagment}
\label{sec:Risikomanagment}

\subsection{Projektspezifische Risiken}

\begin{table}[H]
\begin{tabularx}{\textwidth}{l|>{\raggedright\arraybackslash}X}
\multicolumn{2}{l}{\textbf{R1: Performance der \ac{PCAP}-Library reicht nicht aus}} \\
\hline
Beschreibung & Die \ac{PCAP}-Library ist nicht genügend performant um alle Pakete aufzuzeichnen, es kommt zu Paket-Verlusten die nichts mit der Verbindung zu tun haben. \\
\hline
Massnahme & Wir führen bereits vor der definitiven Wahl der Library Performance Tests durch und können so feststellen, ob sie unseren Ansprüchen (300mbit/s Peak) genügt. \\
\hline
Vorgehen beim Eintreffen & Eine andere \ac{PCAP}-Library wählen.\\
\end{tabularx}
\caption{Risiko - Performance der \ac{PCAP}-Library reicht nicht}
\end{table}

\begin{table}[H]
\begin{tabularx}{\textwidth}{l|>{\raggedright\arraybackslash}X}
\multicolumn{2}{l}{\textbf{R2: Programmiersprache ist schwer zu erlernen}} \\
\hline
Beschreibung & Die für das Projekt gewählte Programmiersprache ist schwer zu erlernen. Oder es sind nicht genügend Mittel (Tutorials, Anleitungen) vorhanden um die Programmiersprache in nützlicher Frist und nicht nach dem \enquote{Trial \& Error}-Verfahren zu erlernen.\\
\hline
Massnahme & Zu Beginn des Projekts möglichst Früh einen Prototyp des Tools entwickeln um so ein Gefühl für die generelle Machbarkeit des Projekts zu erhalten.\\
\hline
Vorgehen beim Eintreffen & Da in diesem Fall das Projekt bereits voll im Gange ist, kommen nur noch Programmiersprachen in Frage von denen bereits gute Kenntnisse vorhanden sind. Konkret kämen bei uns Java oder C\# in Frage.\\
\end{tabularx}
\caption{Risiko - Programmiersprache zu schwer zu erlernen}
\end{table}

\begin{table}[H]
\begin{tabularx}{\textwidth}{l|>{\raggedright\arraybackslash}X}
\multicolumn{2}{l}{\textbf{R3: Ungenügende oder inkorrekte Informationen vom Industriepartner}} \\
\hline
Beschreibung & Die vom Industriepartner zur Verfügung gestellten Informationen bezüglich der Beschaffenheit ihrer Infrastruktur stellen sich als ungenau oder inkorrekt heraus.\\
\hline
Massnahme & Sitzungsprotokoll führen um etwas in der Hand zu haben falls dieser Fall eintritt. \\
\hline
Vorgehen beim Eintreffen & Anhand des Sitzungsprotokoll nachvollziehen, was genau besprochen wurde.\\
\end{tabularx}
\caption{Risiko - Ungenügende oder inkorrekte Informationen}
\end{table}