\section{Qualitätsmassnahmen}
\label{sec:Qualitätsmassnahmen}

\subsection{Issue Verwaltung}
Für die Issue Verwaltung und Zeiterfassung wird YouTrack von Jetbrains verwendet. Es erlaubt Issues zu erfassen, diese gewissen Meilensteinen zuzuordnen und auch jeweils die entsprechende Arbeitszeit zu verbuchen. Ausserdem können automatische Reports wie zum Beispiel ein "Burn-Down-Chart" generiert werden. So hat man den Projektablauf jeder Zeit im Blick. Wir haben YouTrack so konfiguriert dass es nahtlos mit unseren GitHub-Repositories zusammenarbeitet. So können wir z.B. mit einem Commit den Status von einem Issue direkt anpassen.

\subsection{Versionskontrolle}
Für dieses Projekt wurde eine Organisations "IPSecDiagTool" auf Github.com erstellt. Diese Organisation unterhält 4 Repositories. Zwei persönliche Sandbox-Repositories zum Experimentieren und evaluieren der verschiedenen Pcap-Bibliotheken. Ein Haupt-Repository für das \tool . In diesem Repository wird stets nur vollständig funktionsfähiger Code eingecheckt der dann mit einer sinnvollen Commit-Nachricht dokumentiert wird. Separate Features werden in diesem Repository aber in unterschiedlichen Branches entwickelt.
Im vierten Repository befindet sich diese Dokumentation als LaTex Source-Dateien. Auch in dieses Repository wird nur Dokumentations-Code eingepflegt der kompiliert und durch eine verständliche Commit-Nachricht dokumentiert ist.

\subsection{Dokumentation}
Die Dokumentation wird während der Arbeit geschrieben und nicht in eine separaten Phase am Ende. So wird sichergestellt das wirklich unser gesamtes KnowHow erhalten bleibt.
Es werden regelmässige Dokumentations-Reviews geplant um die Qualität der Dokumentation sicher zu stellen.

\subsection{Code Richtlinien}
Da die Programmiersprache noch nicht feststeht haben wir uns noch auf keine spezifischen Richtlinien festgelegt. Dieser Teil wird später noch angepasst. TODO
Grundsätzlich soll der Code gut lesbar sein, mit sauberen und aussagekräftigen Variablen-, Methoden-, und Klassennamen. Alle Kommentare im Code sollen sind auf Englisch zu verfassen.

\subsection{Tests}
Tests sind abhängig von der gewählten Programmiersprache. TODO