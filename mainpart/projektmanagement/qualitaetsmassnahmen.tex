\section{Qualitätsmassnahmen}
\label{sec:Qualitätsmassnahmen}

\subsection{Issue Verwaltung}
Für die Issue Verwaltung und Zeiterfassung wird YouTrack von Jetbrains verwendet. Es erlaubt Issues zu erfassen, diese gewissen Meilensteinen zuzuordnen und auch jeweils die entsprechende Arbeitszeit zu verbuchen. Ausserdem können automatische Reports wie zum Beispiel ein "Burn-Down-Chart" generiert werden. So hat man den Projektablauf jeder Zeit im Blick. Wir haben YouTrack so konfiguriert dass es nahtlos mit unseren GitHub-Repositories zusammenarbeitet. So können wir z.B. mit einem Commit den Status von einem Issue direkt anpassen.

\subsection{Versionskontrolle}
Für dieses Projekt wurde eine Organisations "IPSecDiagTool" auf Github.com erstellt. Diese Organisation unterhält 4 Repositories. Zwei persönliche Sandbox-Repositories zum Experimentieren und evaluieren der verschiedenen Pcap-Bibliotheken. Ein Haupt-Repository für das \tool . In diesem Repository wird stets nur vollständig funktionsfähiger Code eingecheckt der dann mit einer sinnvollen Commit-Nachricht dokumentiert wird. Separate Features werden in diesem Repository aber in unterschiedlichen Branches entwickelt.
Im vierten Repository befindet sich diese Dokumentation als LaTex Source-Dateien. Auch in dieses Repository wird nur Dokumentations-Code eingepflegt der kompiliert und durch eine verständliche Commit-Nachricht dokumentiert ist.

\subsection{Dokumentation}
Die Dokumentation wird während der Arbeit geschrieben und nicht in eine separaten Phase am Ende. So wird sichergestellt das unser gesamtes KnowHow erhalten bleibt.
Es werden regelmässige Dokumentations-Reviews geplant um die Qualität der Dokumentation sicher zu stellen.

Ausserdem werden wir den Code-Dokumentation-Richtlinien von Google zur Verwendung mit Golang folgen. Dies hat den Vorteil dass via godoc zu jeder Zeit, automatisiert eine komplette Dokumentation des Codes erzeugt werden kann. Zudem ist die Dokumentation im Code integriert so dass auch beim direkten lesen des Codes unsere Gedanken-Vorgänge anschaulich sind.

%TODO: Beispiel später mit besserem Code ersetzen.
\begin{lstlisting}[language=go]
//LiveCapture captures all tcp & port 80 packets on eth0.
func LiveCapture() {
	if handle, err := pcap.OpenLive("eth0", 1600, true, 0); err != nil {
		panic(err)
	} else if err := handle.SetBPFFilter("tcp and port 80"); err != nil {
		panic(err)
	} else {
		packetSource := gopacket.NewPacketSource(handle, handle.LinkType())
		for packet := range packetSource.Packets() {
			//Handling packets here
			fmt.Println(packet.Dump())
		}
	}
}
\end{lstlisting}

%TODO: Screenshot von godoc erstellen und einfügen.

\subsection{Code Richtlinien}
Grundsätzlich soll der Code gut lesbar sein, mit sauberen und aussagekräftigen Variablen-, Methoden-, und Klassennamen. Alle Kommentare im Code sind auf Englisch zu verfassen und sollen gem. den Google Dokumentation-Richtlinien erstellt werden (siehe oben).
Änderungen am Code sollen regelmässig in unser Repository auf GitHub committed werden so dass man sie besser nachvollziehen kann.

Da Golang für uns eine neue Sprache ist und wir uns zum Zeitpunkt dieser Textstelle damit noch nicht gut auskennen gelten überall wo möglich die von Google Inc. verwendeten Code-Richtlinien.

%TODO: Code-Richtlinien eventuell noch erweitern.

\subsection{Tests}
Zu jeder erstellten Funktion sollen passenden Unit-Tests geschrieben werden um alle wichtigen Verhaltensweisen zu überprüfen. Die Unit-Tests werden gem. den Golang Vorlagen im selben Ordner wie der zu-testende Code gespeichert. Und zwar nach dem nachfolgenden Schema.

\begin{lstlisting}[language=bash]
bin/
pkg/
src/
    hsr.ipsecdiagtool/
		main/
	    		main.go               # command source
		capture/
	    		capture.go                # command source
	    		capture_test.go           # test source
\end{lstlisting}