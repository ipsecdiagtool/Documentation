\section{Einführung}
\label{sec:Einführung}

\subsection{Zweck}
Dieses Dokument dient zur Planung der Semester/Bachelorarbeit \enquote{\tool{}}. Hier werden die Aufgabenstellung, die Iterationsplanung und die Meilensteine definiert.

\subsection{Aufgabenstellung}
Die Firma \osag{} betreibt im Auftrag ihrer Kunden rund um die Uhr ein weltweites Netz von VPN Verbindungen. Dabei steht eine hohe Qualität und Verfügbarkeit der \ipsec Tunnels an erster Stelle.
Unter dem Linux-Betriebssystem soll ein Diagnose-Tool für \ipsec Verbindungen erstellt werden das folgende Fähigkeiten hat:

Die Programmier- oder Skriptsprache, mit der das Tool erstellt wird ist offen. Es muss aber die Möglichkeit bestehen, die pcapLibrary zum passiven Aufzeichnen der Netzwerkpakete einzubinden.

\begin{itemize}
	\item Passives Bestimmen der \ipsec{} Paket-Verluste durch Erfassen der laufenden \ac{ESP} Sequenznummern von ankommenden \ipsec{} Paketen.
	\item Aktive Diagnose von IPsec Fragmentierungsproblemen und Bestimmung der optimalen MTU. Durch einen Tunnel werden IP Pakete variabler Grösse z.B. in der Form von \ac{ICMP} Requests gesendet und überprüft, ob \acs{IP} Fragmentierung auftritt und Fragmente verloren gehen.
\end{itemize}

\subsection{Ziele}

\begin{itemize}

  \item Auswahl einer geigneten Programmiersprache und Wrappers für die pcapLibrary.
  \item Einarbeiten in \ipsec{} Tunnels und Aufsetzen einer geeigneten Testumgebung.
  \item Spezifikation, Implementation und Test der \tool{} Funktionalität.

\end{itemize}

\subsection{Abgabetermine}
Der offizielle Abgabetermin für die Semesterarbeit ist der Freitag, 29. Mai 2015.
Der späteste Abgabetermin für die Bachelorarbeit ist der Freitag, 12. Juni 2015.
Da diese Arbeit eine Kombination von Bachelor- und Semesterarbeit ist, werden beide Arbeiten am 12. Juni 2015 mit einem jeweils passenden Titelblatt eingereicht. 