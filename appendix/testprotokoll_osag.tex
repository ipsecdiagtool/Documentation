\chapter{Testprotokoll}
\label{chap:Testprotokoll}

\section{Durchführung}
\textbf{Datum:} \\
\textbf{Ort:} \\
\textbf{Getestete Version (GitHub Hash):} \\

\section{Testfälle}

% \Square = leer, \XBox = mit X, \CheckedBox = mit Häckchen
\subsection{Allgemein}
\begin{itemize}
\item[\Square] \textbf{Kompilierung und Deployment:}\\
Das \tool lässt sich auf dem Test-System kompilieren und am erwarteten Speicherort (/opt/ipsecdiagtool/) deployen.
			   
\item[\Square] \textbf{Konfiguration finden:} \\
Das Konfigurationsfile wird am korrekten Ort (/opt/ipsecdiagtool/etc/ipsecdiagtool.conf) erstellt und kann im System gefunden werden. 
			   
\item[\Square] \textbf{Konfiguration anpassen:} \\
Das Konfigurationsfile kann verändert werden und wird beim nächsten Programmstart korrekt eingelesen.
\begin{enumerate} \itemsep1pt \parskip0pt \parsep0pt
  \item Config öffnen, AppID auf 1 setzen.
  \item Das Tool mit dem Debug-Flag starten
  \item Überprüfen ob 1 als AppID ausgegeben wird.
\end{enumerate}

\item[\Square] \textbf{Funktionalität via Help überprüfen:} \\
Das \tool soll via 'ipsecdiagtool help' gestartet werden. Alle aufgelisteten Kommandos und Beispiele sollen ausgeführt werden um zu überprüfen dass sie korrekt aufgelistet wurden.

\item[\Square] \textbf{Daemon Installieren:} \\
Das Tool lässt sich als Daemon installieren und starten.
\begin{enumerate} \itemsep1pt \parskip0pt \parsep0pt
  \item Das Tool starten via 'ipsecdiagtool install'.
  \item SysV wählen.
  \item Überprüfen ob der Daemon verfügbar ist mit 'service ipsecdiagtool start', 'service ipsecdiagtool status', 'service ipsecdiagtool stop'
\end{enumerate}

\item[\Square] \textbf{Daemon Deinstallieren:} \\
Der vom Tool installierte Daemon lässt sich wieder entfernen.
\begin{enumerate} \itemsep1pt \parskip0pt \parsep0pt
  \item Das Tool starten via 'ipsecdiagtool remove'.
  \item SysV wählen.
  \item Überprüfen ob der Daemon deinstalliert wurde mit 'service ipsecdiagtool start'. Wenn ein Fehler kommt wurde er korrekt deinstalliert.
\end{enumerate}
			   
\end{itemize}

\subsection{MTU}
\begin{itemize}
\item[\Square] \textbf{Interaktive MTU Discovery:}\\
Sicherstellen das die MTU Discovery lokal korrekt funktioniert.
\begin{enumerate} \itemsep1pt \parskip0pt \parsep0pt
  \item Bestehende Konfigurationsfiles löschen.  
  \item Das Tool als Daemon installieren 'ipsecdiagtool install' und starten 'start ipsecdiagtool'.
  \item Das Tool lokal starten via 'ipsecdiagtool i mtu'.
  \item Das Tool sollte nun die in der Konfiguration eingestellte SnapLen als MTU melden.
\end{enumerate}
			   
\item[\Square] \textbf{MTU Discovery für ein Tunnel:}\\
Das Tool soll die MTU eines echten IPSec Tunnels ermitteln.
\begin{enumerate} \itemsep1pt \parskip0pt \parsep0pt
  \item Bestehende Konfigurationsfiles neu erstellen oder anpassen so dass folgende Einstellungen konfiguriert sind: SyslogServer, SourceIP, DestinationIP. Die restlichen Einstellungen könne auf den Default Einträgen belassen werden.
  \item Das Tool an beiden Enden des Tunnels als Daemon installieren 'ipsecdiagtool install' und starten 'service ipsecdiagtool start'.
  \item Auf einer Seite des Tunnels die MTU Discovery mit dem lokalen Kommando 'ipsecdiagtool mtu' starten.
  \item Die MTU Discovery wird nun via Tunnel durchgeführt. Das erfolgreiche Resultat sollte auf dem Syslog-Server ersichtlich sein.
\end{enumerate}

\item[\Square] \textbf{MTU Discovery für mehrere Tunnels:}\\
Das Tool soll die MTU von mehreren IPSec Tunnels gleichzeitig feststellen.
\begin{enumerate} \itemsep1pt \parskip0pt \parsep0pt
  \item Bestehende Konfigurationsfiles neu erstellen oder anpassen so dass folgende Einstellungen für 2 oder mehr Tunnels konfiguriert sind: SyslogServer, SourceIP, DestinationIP. Die restlichen Einstellungen könne auf den Default Einträgen belassen werden.
  \item Das Tool an beiden Enden des Tunnels als Daemon installieren 'ipsecdiagtool install' und starten 'service ipsecdiagtool start'.
  \item Auf einer Seite des Tunnels die MTU Discovery mit dem lokalen Kommando 'ipsecdiagtool mtu' starten.
  \item Die MTU Discovery wird nun für alle Tunnels durchgeführt. Das erfolgreiche Resultat sollte auf dem Syslog-Server ersichtlich sein.
\end{enumerate}

\item[\Square] \textbf{MTU Discovery mit unterschiedlicher Anzahl Paketen:}\\
Das Tool soll die gleiche MTU mit einer unterschiedlichen Anzahl an Paketen feststellen.
\begin{enumerate} \itemsep1pt \parskip0pt \parsep0pt
  \item Bestehende Konfigurationsfiles neu erstellen oder anpassen so dass folgende Einstellungen konfiguriert sind: SyslogServer, SourceIP, DestinationIP, ConcurrentPackets. Die restlichen Einstellungen könne auf den Default Einträgen belassen werden.
  \item Das Tool an beiden Enden des Tunnels als Daemon installieren 'ipsecdiagtool install' und starten 'service ipsecdiagtool start'.
  \item Auf einer Seite des Tunnels die MTU Discovery mit dem lokalen Kommando 'ipsecdiagtool mtu' starten.
  \item Nach einem erfolgreichen Durchlauf soll der in "ConcurrentPackets" eingestellte Wert geändert werden. (z.B. in 10er Schritten erhöhen). Die jeweils festgestellte MTU soll aufgeschriben werden.
    \item Am Schluss sollen die notierten MTU Werte verglichen werden. Der Test ist erfolgreich wenn bei 5 Tests nicht mehr als eine Abweichung auftritt.
\end{enumerate}

\end{itemize}
\subsection{Packetloss}
\begin{itemize}
\item[\Square] \textbf{Interaktive Packetloss Discovery:}\\
Sicherstellen das ESP Pakete verarbeitet werden.
\begin{enumerate} \itemsep1pt \parskip0pt \parsep0pt
  \item Bestehende Konfigurationsfiles anpassen.  
  \item Das Tool als Daemon installieren 'ipsecdiagtool install' und starten 'start ipsecdiagtool'.
  \item Das Tool lokal starten via 'ipsecdiagtool i packetloss'.
  \item Das Tool sollte nun ankommende und ausgehende ESP-Packete anzeigen.
\end{enumerate}

\begin{itemize}
\item[\Square] \textbf{Packetloss via PCAP-File feststellen:}\\
Sicherstellen, ob Packete eines PCAP-Files korrekt verarbeitet und der Packetloss korrekt angezeigt wird.
\begin{enumerate} \itemsep1pt \parskip0pt \parsep0pt
  \item Bestehende Konfigurationsfiles anpassen.  
  \item Das Tool als Daemon installieren 'ipsecdiagtool install' und starten 'start ipsecdiagtool'.
  \item Das Tool lokal starten via 'ipsecdiagtool i packetloss'.
  \item Das Tool sollte nun einen Eintrag im Syslog Server erstellen mit der korrekten Meldung über den Verlust von Paketen. Gleichzeitig sollte ein CSV-File mit den Verlorenen Paketen angelegt werden.
\end{enumerate}

\begin{itemize}
\item[\Square] \textbf{Packetloss via live Capture feststellen}\\
Sicherstellen, ob Packete bei live capturing verarbeitet und der Packetloss korrekt angezeigt wird.
\begin{enumerate} \itemsep1pt \parskip0pt \parsep0pt
  \item Bestehende Konfigurationsfiles anpassen.  
  \item Das Tool als Daemon installieren 'ipsecdiagtool install' und starten 'start ipsecdiagtool'.
  \item Das Tool lokal starten via 'ipsecdiagtool i packetloss'.
  \item Das Tool sollte nun einen Eintrag im Syslog Server erstellen mit der korrekten Meldung über den Verlust von Paketen. Gleichzeitig sollte ein CSV-File mit den Verlorenen Paketen angelegt werden.
\end{enumerate}

\end{itemize}