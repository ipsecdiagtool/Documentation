\chapter{Testprotokoll}
\label{chap:Testprotokoll}

\section{Durchführung}
\textbf{Datum:} \\
\textbf{Ort:} \\
\textbf{Getestete Version (GitHub Hash):} \\

\section{Testfälle}

% \Square = leer, \XBox = mit X, \CheckedBox = mit Häckchen
\subsection{Allgemein}
\begin{itemize}
\item[\Square] \textbf{Kompilierung und Deployment:}\\
Das \tool lässt sich auf dem Test-System kompilieren und am erwarteten Speicherort (/opt/ipsecdiagtool/) deployen.
			   
\item[\Square] \textbf{Konfiguration finden:} \\
Das Konfigurationsfile wird am korrekten Ort (/opt/ipsecdiagtool/etc/ipsecdiagtool.conf) erstellt und kann im System gefunden werden. 
			   
\item[\Square] \textbf{Konfiguration anpassen:} \\
Das Konfigurationsfile kann verändert werden und wird beim nächsten Programmstart korrekt eingelesen.
\begin{enumerate} \itemsep1pt \parskip0pt \parsep0pt
  \item Config öffnen, AppID auf 1 setzen.
  \item Das Tool mit dem Debug-Flag starten
  \item Überprüfen ob 1 als AppID ausgegeben wird.
\end{enumerate}

\item[\Square] \textbf{Daemon Installieren:} \\
Das Tool lässt sich als Daemon installieren und starten.
\begin{enumerate} \itemsep1pt \parskip0pt \parsep0pt
  \item Das Tool starten via 'ipsecdiagtool install'.
  \item Überprüfen ob der Daemon verfügbar ist mit 'start ipsecdiagtool', 'status ipsecdiagtool', 'stop ipsecdiagtool'
\end{enumerate}

\item[\Square] \textbf{Daemon Deinstallieren:} \\
Der vom Tool installierte Daemon lässt sich wieder entfernen.
\begin{enumerate} \itemsep1pt \parskip0pt \parsep0pt
  \item Das Tool starten via 'ipsecdiagtool remove'.
  \item Überprüfen ob der Daemon deinstalliert wurde mit 'start ipsecdiagtool'. Wenn ein Fehler kommt wurde er korrekt deinstalliert.
\end{enumerate}
			   
\end{itemize}

\subsection{MTU}
\begin{itemize}
\item[\Square] \textbf{Kompilierung und Deployment:}\\
Das \tool lässt sich auf dem Test-System kompilieren und am erwarteten Speicherort (/opt/ipsecdiagtool/) deployen.
		

\end{itemize}
\subsection{Packetloss}
